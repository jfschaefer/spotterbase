\documentclass{article}
\usepackage{amsthm,amsmath,amsfonts}
\newtheorem{definition}{Definition}
\newtheorem{theorem}{Theorem}
\newtheorem{corollary}{Corollary}
\author{Nota Mathematician}
\title{Test Document: There are interesting triangles}
\begin{document}
\maketitle
\section{Introduction}
In this paper, we will prove that interesting triangles exist in two-dimensional Euclidean space.

\begin{definition}
    A \textbf{triangle in $n$-dimensional Euclidean space} is a set $T$ of cardinality 3, where $P \in \mathbb{R}^n$ for all $P \in T$.
\end{definition}

\begin{definition}
    We call a triangle $T = \{P_1, P_2, P_3\}$ \textbf{interesting} iff there is no $t \in \mathbb{R}$ such that $P_1 + t (P_2-P_1) = P_3$.
\end{definition}

Note that interestingness is well-defined (does not depend on the ``order'' of $P_1$, $P_2$ and $P_3$).
The proof of this is left as an exercise.

\begin{theorem}
    Both $T_1 := \{(0, 0), (1, 0), (1, 1)\}$ and $T_2 := \{(1, -1), (0, 1), (1, 0)\}$ are interesting triangles.
\end{theorem}

\begin{proof}
    Trivial.
\end{proof}

\begin{corollary}
    It follows that interesting triangles exist in $\mathbb{R}^2$.
\end{corollary}

\end{document}
