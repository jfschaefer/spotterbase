\documentclass{article}
\usepackage{amsthm,amsmath,amsfonts}
\newtheorem{definition}{Definition}
\newtheorem{theorem}{Theorem}
\newtheorem{corollary}{Corollary}
\author{Nota Physicist}
\title{Test Document: The speed of carbon dioxide absorption in forests}
\begin{document}
\maketitle
\section{Motivation}
One aspect of combating climate change is to study how much carbon dioxide
gets absorbed in different eco systems.
The units involved are typically very difficult to grasp intuitively,
such as $\textrm{kg}\textrm{CO}_2\textrm{ha}^{-1}\textrm{y}^{-1}$.
A forest might, for example, absorb $1500\,\textrm{kg}\textrm{CO}_2\textrm{ha}^{-1}\textrm{y}^{-1}$,
which is a quantity that you cannot easily get a good intuition for.

\section{Absorption as speed}
By switching from the mass of the absorbed $\textrm{CO}_2$ to its volume,
it is straight-forward to compute the absorption as a speed.

For example, assuming the $\textrm{CO}_2$ density to be $\rho = 1.8\,\textrm{kg}\textrm{m}^{-3}$,
we can convert the unintuitive $1500\,\textrm{kg}\textrm{CO}_2\textrm{ha}^{-1}\textrm{y}^{-1}$
to the much more intuitive absorption speed of
$2.64\,\textrm{nm}/\textrm{s}$, which is, of course, rather slow.

\section{Conclusion}
We propose to express $\textrm{CO}_2$ absorption rates of eco systems as a speed
rather than commonly used units like $\textrm{kg}\textrm{CO}_2\textrm{ha}^{-1}\textrm{y}^{-1}$,
which are very unintuitive.
For example, that allows us to specify the absorption speed of a forest as
$2.64\,\textrm{nm}/\textrm{s}$ (or $8.3\,\textrm{cm}$ per year), rather than
$1500\,\textrm{kg}\textrm{CO}_2\textrm{ha}^{-1}\textrm{y}^{-1}$.


\end{document}
